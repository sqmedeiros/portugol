\documentclass{report}

\usepackage[utf8]{inputenc}
\usepackage[brazil]{babel}
\usepackage{lmodern}
\usepackage[T1]{fontenc}
\usepackage{graphicx}
\usepackage{xspace}
\usepackage{amssymb}
\usepackage{amsmath}
\usepackage{amsfonts}
\usepackage{amsthm}
\usepackage{epsfig}

\begin{document}

\thispagestyle{empty}
\begin{center}

\begin{center}
%\tt{LabLua}\\
%\tt{www.lua.inf.puc-rio.br}
\end{center}
\vspace{6cm}
\Huge{Lógica de Programação}\\
\vspace{1.5cm}
\Large{Sérgio Queiroz de Medeiros} \\
\Large{Escola de Ciências e Tecnologia} \\
\Large{UFRN}
\vspace{5cm}
\\
\large{Julho de 2016}
\end{center}

\tableofcontents

\pagebreak

%\textbf{\huge{Prefácio}}

%\vspace{1cm}



\chapter{Introdução}

\section{Tipos Básicos}

\begin{itemize}
	\item \textbf{booleano}
	\item \textbf{inteiro}
	\item \textbf{numero}
	\item \textbf{texto}
\end{itemize}


\section{Declaração de Array}

A forma geral para declarar um array é a seguinte: 
\begin{verbatim}
nomeDoTipo[] identificador
\end{verbatim}

No exemplo acima, apenas declaramos um array, mas não o
inicializamos. Abaixo declaramos e inicializamos um
array de inteiros com 10 elementos:
\begin{verbatim}
inteiro[] x = novo inteiro[10]
\end{verbatim}

O primeiro elemento de $x$ está no índice 1 o último
elemento no índice 10. É um erro acessar um índice do
array fora desse intervalo. Abaixo temos o exemplo de
um programa que inicializa cada posição $i$ de um array
com o $i^2$ e imprime o resultado:
\begin{verbatim}
inteiro[] a

a = novo inteiro[5]

inteiro i = 1
repita enquanto i <= 5
  a[i] = i * i
  i = i + 1
fim

i = 1
repita enquanto i <= 5
  escreva(a[i])
  i = i + 1
fim
\end{verbatim}


Devemos usar um par de \texttt{[]} para declarar cada dimensão
do nosso array, como mostrado abaixo: 
\begin{verbatim}
inteiro[][] x = novo inteiro[5][3]
\end{verbatim}

A expressão \texttt{novo inteiro[5][3]} diz que $x$ é um
array que possui 5 elementos na primeira dimensão, e cada
uma desses elementos possui 3 elementos na segunda dimensão.


\section{Funções Básicas}

\begin{itemize}
	\item \textbf{escreva (parâmetro1, parâmetro2, ..., parâmetroN)}
	
	\item \textbf{leia(parâmetro1, parâmetro2, ..., parâmetroN)}
	
	\item \textbf{inteiro textoComp (texto s)}: retorna o comprimento do texto $s$
	
	\item \textbf{texto textoSub (texto s, inteiro inicio, inteiro fim)}:
	retorna a porção do texto $s$ que começa no índice $inicio$ e termina
  no índice $fim$. O índice $fim$ é opcional, se ele não for fornecido a
  porção do texto $s$ retornada começa em $inicio$ e vai até o fim de $s$.
	
	\item \textbf{texto textoPos (texto s, inteiro inicio)}:
	retorna o texto no índice $inicio$ de $s$
\end{itemize}


\section{Declaração de Função}

Forma geral de declarar uma função: \\
\textbf{funcao} nomeFuncao (tipo1 nome1, ..., tipoN nomeN) \textbf{retorna} tipoRetorno

Caso a função não retorne nenhum valor, a última parte não é necessária e a
função pode ser declarada como: \\
\textbf{funcao} nomeFuncao (tipo1 nome1, ..., tipoN nomeN)


Na figura~\ref{fig:perfeito} temos o exemplo de uma função que
determina se um dado número é perfeito ou não. Um número inteiro
$x$ é perfeito se ele é igual à soma dos seus divisores, excetuando-se
ele mesmo. 

\begin{figure}
\begin{verbatim}
funcao ehPerfeito (inteiro x) retorna booleano
  inteiro soma = 1
  inteiro divisor = 2
  repita enquanto divisor <= x / 2
    se x mod divisor == 0
      soma = soma + divisor
    fim
    divisor = divisor + 1
  fim	

  retorne soma == x
fim

escreva(ehPerfeito(6))
escreva(ehPerfeito(9))
escreva(ehPerfeito(28))
escreva(ehPerfeito(1000))
\end{verbatim}
\caption{Função que Determina se um Número é Perfeito}
\label{fig:perfeito}
\end{figure}

\end{document}


