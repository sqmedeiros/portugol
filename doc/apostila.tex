\documentclass{report}

\usepackage[utf8]{inputenc}
\usepackage[brazil]{babel}
\usepackage{lmodern}
\usepackage[T1]{fontenc}
\usepackage{graphicx}
\usepackage{xspace}
\usepackage{amssymb}
\usepackage{amsmath}
\usepackage{amsfonts}
\usepackage{amsthm}
\usepackage{epsfig}
\usepackage{url}

\newcommand{\Egol}{ECTGol}

\begin{document}

\thispagestyle{empty}
\begin{center}

\begin{center}
%\tt{LabLua}\\
%\tt{www.lua.inf.puc-rio.br}
\end{center}
\vspace{6cm}
\Huge{Lógica de Programação}\\
\vspace{1.5cm}
\Large{Sérgio Queiroz de Medeiros} \\
\Large{Escola de Ciências e Tecnologia} \\
\Large{UFRN}
\vspace{5cm}
\\
\large{Julho de 2016}
\end{center}

\tableofcontents

\pagebreak

%\textbf{\huge{Prefácio}}

%\vspace{1cm}


\chapter{Introdução}

Em um curso como Lógica de Programação, onde o aluno
vai começar o aprendizado de algoritmos e de programção
de computadores, é comum a utilização de uma linguagem
de programação que possa facilitar o entendimento do
assunto, ao invés de usar linguagens de programação
mais convencionais.

Geralmente uma linguagem de programação utiliza termos em inglês,
de modo que ela possa ser usada por pessoas de todo o mundo.
Contudo, o uso de termos em inglês pode dificultar o aprendizado
de alguns alunos. Por conta disso, algumas linguagens, voltadas
principalmente para o aprendizado de programação, usam termos em
português. Essas linguagens são usualmente chamadas de \emph{Portugol}.

Neste curso utilizaremos a linguagem de programação \emph{\Egol},
cuja sintaxe foi inspirada na sintaxe da linguagem Quorum
(\url{https://www.quorumlanguage.com/}).

Com \Egol será possível implementarmos algoritmos que envolvam
estruturas de decisão, estruturas de repetição, coleções de dados
e funções.

Para executarmos os programas escritos em \Egol é necessário
instalar o ambiente de programação da linguagem, que está disponível em
\url{https://github.com/alex7alves/InterfaacePesquisa}.


\section{Programa \emph{Olá, mundo}}

Abaixo temos o nosso primeiro programa \Egol:
%
\begin{verbatim}
escreva("Olá, mundo")
\end{verbatim}
%
No programa acima usamos a função \texttt{escreva},
que recebe um valor e o imprime para a saída padrão
do computador (o monitor, geralmente). Nesse programa
\texttt{escreva} recebeu um texto entre aspas duplas.
Quando executado, esse programa imprime o texto \emph{Olá, mundo}.  

Escreva o programa acima no ambiente de programação de \Egol,
salve o arquivo como \texttt{alo.gol} e tente executá-lo.
Use sempre a extensão \texttt{.gol} para os arquivos com
programas \Egol.

A função \texttt{escreva} não imprime só texto, ela também
imprime valores inteiros e outros tipos de dados que veremos
mais adiante. Por exemplo, no programa a seguir a função
\texttt{escreva} recebe dois valores, um texto e o valor
inteiro 42:
%
\begin{verbatim}
escreva("A resposta é ", 42)
\end{verbatim}
%
Note que usamos uma vírgula para separar os
valores passados para a função \texttt{escreva}.


\section{Variáveis}

Vamos agora escrever um programa que pede para o usuário
digitar dois valores inteiros, calcula a soma desses valores
e imprime o resultado.

Abaixo, temos o programa \texttt{somaInteiro.gol}, que
calcula a soma de valores inteiros digitados pelo usuário:
%
\begin{verbatim}
01  inteiro valor1, valor2, soma
02
03  escreva("Digite um valor inteiro: ")
04  leia(valor1)
05
06  escreva("Digite outro valor inteiro: ")
07  leia(valor2)
08
09  soma = valor1 + valor2
10  escreva("A soma deu ", soma)
\end{verbatim}
%
Na linha 1 declaramos as variáveis \texttt{valor1}, \texttt{valor2}
e \texttt{soma}, do tipo \textbf{inteiro}. Uma variável é um nome associado
a uma região da memória do computador. Podemos usar esse nome para
acessar e alterar o valor armazenado nessa região de memória. Em \Egol
toda variável possui um tipo, que indica a natureza dos valores que
iremos armazenar naquela variável. No acima, o tipo \textbf{inteiro}
indica que iremos guardar valores inteiros na variáveis
\texttt{valor1}, \texttt{valor2} e \texttt{soma}, tais como
42, -255, 0 e 1000000. 

Na linha 3, usamos a função \texttt{escreva}, que vimos anteriormente,
para pedir que o usuário forneça um valor inteiro.

Usamos a função \texttt{leia} na linha 4 para ler a informação digitada
pelo usuário e armazená-la na região de memória associada à variável
\texttt{valor1}. De modo similar, na linha 7 usamos \texttt{leia}
para guardar na variável\texttt{valor2} o segundo valor inteiro
digitado pelo usuário.


Na linha 9 declaramos e inicializamos a variável \texttt{soma},
também do tipo \textbf{inteiro}. Para atribuir um valor a uma
variável usamos o operador \texttt{=}.



Na linha 1 declaramos duas variáveis inteiras
\texttt{valor} e \texttt{valor21



\section{Tipos Básicos}

\begin{itemize}
	\item \textbf{booleano}
	\item \textbf{inteiro}
	\item \textbf{numero}
	\item \textbf{texto}
\end{itemize}



\section{Exercícios}




\section{Declaração de Array}

A forma geral para declarar um array é a seguinte: 
\begin{verbatim}
nomeDoTipo[] identificador
\end{verbatim}

No exemplo acima, apenas declaramos um array, mas não o
inicializamos. Abaixo declaramos e inicializamos um
array de inteiros com 10 elementos:
\begin{verbatim}
inteiro[] x = novo inteiro[10]
\end{verbatim}

O primeiro elemento de $x$ está no índice 1 o último
elemento no índice 10. É um erro acessar um índice do
array fora desse intervalo. Abaixo temos o exemplo de
um programa que inicializa cada posição $i$ de um array
com o $i^2$ e imprime o resultado:
\begin{verbatim}
inteiro[] a

a = novo inteiro[5]

inteiro i = 1
repita enquanto i <= 5
  a[i] = i * i
  i = i + 1
fim

i = 1
repita enquanto i <= 5
  escreva(a[i])
  i = i + 1
fim
\end{verbatim}


Devemos usar um par de \texttt{[]} para declarar cada dimensão
do nosso array, como mostrado abaixo: 
\begin{verbatim}
inteiro[][] x = novo inteiro[5][3]
\end{verbatim}

A expressão \texttt{novo inteiro[5][3]} diz que $x$ é um
array que possui 5 elementos na primeira dimensão, e cada
uma desses elementos possui 3 elementos na segunda dimensão.


\section{Funções Básicas}

\begin{itemize}
	\item \textbf{escreva (parâmetro1, parâmetro2, ..., parâmetroN)}
	
	\item \textbf{leia(parâmetro1, parâmetro2, ..., parâmetroN)}
	
	\item \textbf{inteiro textoComp (texto s)}: retorna o comprimento do texto $s$
	
	\item \textbf{texto textoSub (texto s, inteiro inicio, inteiro fim)}:
	retorna a porção do texto $s$ que começa no índice $inicio$ e termina
  no índice $fim$. O índice $fim$ é opcional, se ele não for fornecido a
  porção do texto $s$ retornada começa em $inicio$ e vai até o fim de $s$.
	
	\item \textbf{texto textoPos (texto s, inteiro inicio)}:
	retorna o texto no índice $inicio$ de $s$
\end{itemize}


\section{Declaração de Função}

Forma geral de declarar uma função: \\
\textbf{funcao} nomeFuncao (tipo1 nome1, ..., tipoN nomeN) \textbf{retorna} tipoRetorno

Caso a função não retorne nenhum valor, a última parte não é necessária e a
função pode ser declarada como: \\
\textbf{funcao} nomeFuncao (tipo1 nome1, ..., tipoN nomeN)


Na figura~\ref{fig:perfeito} temos o exemplo de uma função que
determina se um dado número é perfeito ou não. Um número inteiro
$x$ é perfeito se ele é igual à soma dos seus divisores, excetuando-se
ele mesmo. 

\begin{figure}
\begin{verbatim}
funcao ehPerfeito (inteiro x) retorna booleano
  inteiro soma = 1
  inteiro divisor = 2
  repita enquanto divisor <= x / 2
    se x mod divisor == 0
      soma = soma + divisor
    fim
    divisor = divisor + 1
  fim	

  retorne soma == x
fim

escreva(ehPerfeito(6))
escreva(ehPerfeito(9))
escreva(ehPerfeito(28))
escreva(ehPerfeito(1000))
\end{verbatim}
\caption{Função que Determina se um Número é Perfeito}
\label{fig:perfeito}
\end{figure}

\end{document}


